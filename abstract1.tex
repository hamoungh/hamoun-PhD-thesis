Cloud computing is a flexible platform for software as a service, as more and more applications are deployed on cloud. Major challenges in cloud include how to characterize the workload of the applications and how to manage the cloud resources efficiently by sharing them among many applications. The current state of the art considers a simplified model of the system, either ignoring the software components altogether or ignoring the relationship between individual software services.  This thesis considers the following resource management problems for cloud-based service providers: (i) how to estimate the parameters of the current workload, (ii) how to meet Quality of Service (QoS) targets while minimizing infrastructure cost, (iii) how to allocate resources considering performance costs of virtual machine reconfigurations. To address the above problems, we propose a model-based feedback loop approach. The cloud infrastructure, the services, and the applications are modelled using Layered Queuing Models (LQM). These models are then optimized. Mathematical techniques are used to reduce the complexity of the models and address the scalability issues. The main contributions of this thesis are:  (i) Extended Kalman Filter (EKF) based techniques improved by dynamic clustering for scalable estimation of workload parameters, (ii) combination of adaptive empirical models (tuned during runtime) and stepwise optimizations for improving the overall allocation performance, (iii) dynamic service placement algorithms that consider the cost of virtual machine reconfiguration  


